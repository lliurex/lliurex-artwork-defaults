\documentclass[11pt]{article}
%Gummi|061|=)
\usepackage[utf8]{inputenc}
\usepackage[spanish]{babel}
\title{\textbf{Benvinguts al Gummi 0.6.1}}
\author{Alexander van der Mey\\
		Wei-Ning Huang\\
		Dion Timmermann}
\date{}
\begin{document}

\maketitle

\section{Abans de començar}

Este és el Gummi 0.6.1. S'han afegit moltes característiques excitants a la versió 0.6. L'editor de documents ara es troba en una pestanya i això permet treballar en múltiples documents simultàniament. Utilitzant el nou menú de projectes, podeu agrupar fitxers per a accedir-hi fàcilment. 

A més, s'ha afefit suport per a dos sistemes de construcció de {\LaTeX} d'alt nivell \emph{rubber}\footnote{https://launchpad.net/rubber/}  i \emph{latexmk}\footnote{http://www.phys.psu.edu/{\textasciitilde}collins/software/latexmk-jcc/}. Podeu configurar el vostre componedor preferit en la pestanya de compilació del menú de preferències. Els componedors que no s'instal·len en el sistema no es podran seleccionar. 

També s'hi ha afegit un mode de previsualització contínua de PDF. Este mode està habilitat per defecte, però es pot inhabilitar a través del menú \emph{(Visualitza $\rightarrow$ Disposició de la pàgina en la previsualització)}. Una característica complementària de l'anterior és la integració del SyncTeX, que vos permet sincronitzar la posició en l'editor amb la que apareix en la previsualització del PDF. 

\section{Retroacció}
Esperem que gaudiu utilitzant esta versió tant com hem gaudit creant-la. Si teniu cap comentari, suggeriment o desitgeu informar d'un problema, podeu posar-vos en contacte a: \emph{http://gummi.midnightcoding.org}.

\section{Una cosa més}
El text per defecte antic s'ha emmagatzemat com a plantilla. Podeu utilitzar el menú de plantilla per a accedir-hi i restaurar-lo. 

\end{document}